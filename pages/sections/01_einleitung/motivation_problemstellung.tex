\label{sec:motivation_problemstellung}
Durch die immer größer werdende Globalisierung werden immer mehr Güter verschickt. Zudem steigt ebenso die Nachfrage an transparenten und nachverfolgbaren Transportwegen um die Qualität und Sicherheit der Ware zu gewährleisten. Dies stellt Transportunternehmen vor eine große Herausforderung da die Überwachung zum einen oft über eine sehr große Strecke sowie über Ländergrenzen bewerkstelligt werden muss, zum anderen gibt es eine Vielzahl an zu überwachenden Sendungen und damit an benötigten Trackern.

Traditionelle Tracking-Systeme, die auf Mobilfunktechnologien oder Satellitenkommunikation basieren, sind oft mit hohen Betriebskosten, einem hohen Energieverbrauch oder einer eingeschränkten Reichweite konfrontiert. Dies führt zu Sichtbarkeitslücken in der Lieferkette, was wiederum Risiken wie Diebstahl, Verlust oder Beschädigung von Gütern erhöht.

Vor diesem Hintergrund bietet die Low-Power Wide-Area Network (LPWAN)-Technologie, insbesondere Long-Range Wide-Area-Network (LoRaWAN), vielversprechende Ansätze. LoRaWAN zeichnet sich durch seine Fähigkeit aus, Daten über große Distanzen bei geringem Energieverbrauch zu übertragen. Das macht diese Technologie ideal für Anwendungen im Bereich des Internet of Things (IoT) und der Logistik. Trotz des Potenzials von LoRaWAN für Tracking-Anwendungen, gibt es noch Forschungsbedarf hinsichtlich der praktischen Implementierung und der Analyse der Verbindungsqualität unter realen Bedingungen.