Ziel dieser Arbeit ist die Entwicklung eines modularen Tracking-Geräts, das die in Unterkapitel \ref{sec:motivation_problemstellung} beschriebenen Herausforderungen adressiert.
Das Gerät soll flexibel an unterschiedliche Anforderungen der Transportlogistik angepasst werden können, um in einer Vielzahl von Einsatzszenarien abzudecken.
Des Weiteren soll ein Analysetool entwickelt werden, das mithilfe verschiedener Methoden zur Vorhersage der Verbindungsqualität und der damit verbundenen Netzwerkabdeckung den Transportweg möglichst transparent abbildet.
Ziel ist es, potenzielle Signalverluste und Sendeverzögerungen frühzeitig vorherzusagen, um die Zuverlässigkeit der Datenübertragung entlang der Route besser bewerten zu können.