Zur Bewertung der Verbindungsqualität werden primär drei Schlüsselmetriken herangezogen: der Received Signal Strength Indicator (RSSI), das Signal-to-Noise Ratio (SNR) und die Packet Delivery Rate (PDR).

Der \textbf{RSSI} ist wie im Namen schon enthalten ein Indikator für das empfangene Signal. Er dient als relatives Maß und gibt die Stärke des Funksignals an. Er wird typischerweise in dBm ausgedrückt, wobei Werte näher an Null ein stärkeres empfangenes Signal bedeuten. Faktoren, die den RSSI hauptsächlich beeinflussen, sind der Pfadverlust oder der Antennengewinn. Zudem kommen auch noch weitere Verluste durch Kabel und Steckverbinder, für die Antenne, an Sender und Empfänger hinzu. Darüber hinaus kann der RSSI aufgrund von Signalkollisionen, externen Interferenzen und dem Nah-Fern-Effekt abnehmen. Als direktes Maß für die empfangene Leistung ist der RSSI eine entscheidende Eingangsgröße für viele Ausbreitungsmodelle. 

Oft als S/N bezeichnet, stellt das \textbf{SNR} das Verhältnis der empfangenen Signalleistung zum Umgebungsrauschpegel dar, üblicherweise in Dezibel (dB) ausgedrückt. Es ist eine kritische Metrik zur Bestimmung der Gesamtqualität des empfangenen Signals. \autocite{SignaltoNoiseRatioSNR2020} Das SNR wird mit der Formel \ref{eq:snr} berechnet.
\begin{equation}
\label{eq:snr}
SNR (dB) = P_{receivedSignal} (dBm) - P_{noise} (dBm)
\end{equation}


Auch als Packet Reception Rate (PRR) bezeichnet, quantifiziert die \textbf{PDR} den Prozentsatz der Datenpakete, die erfolgreich über eine Kommunikationsverbindung gesendet und empfangen werden. Sie dient als Maß für die Netzwerkkonsistenz und -leistung. Die PDR wird durch eine Kombination von Faktoren beeinflusst, darunter die Bandbreite, die Übertragungsdistanz und die zugrunde liegende Signalqualität (RSSI, SNR). Insbesondere niedrige RSSI-Werte, insbesondere unter einem bestimmten Schwellenwert (z. B. -120 dBm), können zu Paketkorruption oder vollständigem Verlust führen.

Während der RSSI eine Angabe der Rohsignalstärke liefert, bietet das SNR ein nuancierteres Maß für die Signalqualität relativ zum Rauschen. Letztendlich stellt die PDR die Ergebnisgröße dar, die die kombinierten Effekte von Signalstärke, Signalqualität und vorherrschenden Netzwerkbedingungen zusammenfasst. \autocite{RSSISNR}