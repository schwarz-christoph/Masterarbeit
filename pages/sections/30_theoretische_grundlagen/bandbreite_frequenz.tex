Die \textbf{Frequenz} beschreibt, wie oft sich ein periodisches Signal pro Sekunde wiederholt. Sie wird in Hertz (Hz) angegeben und ist eng mit der Wellenlänge $\lambda$ verknüpft. Diese lässt sich aus der Ausbreitungsgeschwindigkeit $c$ des Signals und der Frequenz $f$ über $\lambda = c / f$ berechnen. Höhere Frequenzen entsprechen kürzeren Wellenlängen und erfordern in der Regel kleinere Antennenabmessungen.

Die \textbf{Bandbreite} eines Kommunikationskanals bezeichnet den Frequenzbereich, innerhalb dessen ein Signal mit akzeptabler Qualität übertragen werden kann. Sie wird in Hertz angegeben und ist sowohl durch physikalische Eigenschaften des Übertragungsmediums als auch durch elektronische Komponenten begrenzt. Die Bandbreite bestimmt zusammen mit der Signalqualität die maximal mögliche Datenrate. In vielen Kanälen beträgt sie nur einen Bruchteil der Mittenfrequenz, typischerweise zwischen $1\%$ und $10\%$ des Trägerfrequenzbereichs. \autocite[S. 4]{proakisDigitalCommunications2008}
