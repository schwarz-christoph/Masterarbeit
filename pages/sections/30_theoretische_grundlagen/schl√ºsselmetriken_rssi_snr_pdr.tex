Zur Bewertung der Verbindungsqualität werden primär drei Schlüsselmetriken herangezogen: der Received Signal Strength Indicator (RSSI), das Signal-to-Noise Ratio (SNR) und die Packet Delivery Rate (PDR).

Der RSSI ist ein Indikator für das empfangene Signal. Er dient als relatives Maß und gibt die Stärke des Funksignals an. Er wird typischerweise in Dezibel bezogen auf ein Milliwatt (dBm) ausgedrückt, wobei Werte näher an Null ein stärkeres empfangenes Signal bedeuten. Faktoren, die den RSSI hauptsächlich beeinflussen, sind ein Pfadverlust oder ein Antennengewinn. Pfadverlust bezeichnet die Abschwächung eines Funksignals auf dem Weg von der Sende- zur Empfangsantenne. Er entsteht durch die Entfernung zwischen den Geräten sowie durch Hindernisse, Reflexionen, Beugungen und Streuungen in der Umgebung. Ein hoher Pfadverlust führt dazu, dass die Signalstärke am Empfänger stark reduziert wird. Ebenso sinkt die Wahrscheinlichkeit einer fehlerfreien Übertragung. Antennengewinn beschreibt dagegen die Fähigkeit einer Antenne, die abgestrahlte oder empfangene Energie in eine bestimmte Richtung zu bündeln. Ein höherer Gewinn bedeutet, dass die abgestrahlte Leistung in der Austrahlrichtung größer ist und somit die Reichweite oder Empfindlichkeit in dieser Richtung steigt. Zusätzlich entstehen jedoch ebenso Verluste durch Kabel und Steckverbinder der Antenne, die sowohl Sender als auch Empfänger beeinflussen. Darüber hinaus kann die Sendequalität durch den Nah-Fern-Effekt abnehmen. Der Nah-Fern-Effekt tritt in Funknetzen auf, wenn ein Empfänger gleichzeitig Signale von mehreren Sendern empfängt, die stark unterschiedliche Empfangsleistungen haben. Ein naher Sender mit starkem Signal kann den Empfänger übersteuern oder die Signalerkennung erschweren, sodass ein weiter entfernter Sender mit schwachem Signal nicht mehr zuverlässig empfangen wird. Als direktes Maß für die empfangene Leistung ist der RSSI eine entscheidende Eingangsgröße für viele Ausbreitungsmodelle zm Beispiel das Log-Distance Path Loss Model. 

Das SNR stellt das das Verhältnis der empfangenen Signalleistung zum Umgebungsrauschpegel dar, üblicherweise in Dezibel (dB) ausgedrückt. Es ist eine kritische Metrik zur Bestimmung der Gesamtqualität des empfangenen Signals \autocite{SignaltoNoiseRatioSNR2020}. Das SNR wird mit der Formel \ref{eq:snr} berechnet.
\begin{equation}
\label{eq:snr}
SNR (dB) = P_{receivedSignal} (dBm) - P_{noise} (dBm)
\end{equation}


Auch als Packet Reception Rate (PRR) bezeichnet, quantifiziert die \textbf{PDR} den Prozentsatz der Datenpakete, die erfolgreich über eine Kommunikationsverbindung gesendet und empfangen werden. Sie dient als Maß für die Netzwerkkonsistenz und -leistung. Die PDR wird durch eine Kombination von Faktoren beeinflusst, darunter die Bandbreite, die Übertragungsdistanz und die zugrunde liegende Signalqualität (RSSI, SNR). Insbesondere niedrige RSSI-Werte, insbesondere unter einem bestimmten Schwellenwert (beispielsweise -120 dBm), können zu Paketkorruption oder vollständigem Verlust führen.

Während der RSSI eine Angabe der Rohsignalstärke liefert, bietet das SNR ein Maß für die Signalqualität relativ zum Rauschen. Letztendlich stellt die PDR die Ergebnisgröße dar, die die kombinierten Effekte von Signalstärke, Signalqualität und vorherrschenden Netzwerkbedingungen zusammenfasst \autocite{RSSISNR}.