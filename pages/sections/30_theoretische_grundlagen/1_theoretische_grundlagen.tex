Im nachfolgenden Teil sollen die theoretische Grundlagen erläutert werden die für das Verständis der nachfolgenden Arbeit wichtig sind. Dabei bilden die Prinzipien der drahtlosen Datenübertragung das Fundament, auf dem alle weiteren technischen Konzepte aufbauen.
\label{kap:grundlagen}
\subsection{Drahtlose Datenübertragung}
\label{sec:drahtlosedatenübertragung}
Drahtlose Datenübertragungverfahren ermöglichen die Übertragung von Informationen ohne elektrische Leiter. Sie nutzen elektromagnetische Wellen (Radiowellen), Magenetelder und elektrische Felder als Übertragungsmedium und können somit eine Kommunikation über Entfernungen von mehreren Kilometern oder mehr ermöglichen. 

Dieses Grundlegende Prinzip ermöglicht Anwendungen die deutlich portabler und flexiebler sind als kabelgebunden Verbindungen. Der Kernmechanismuss besteht darin die am Sender als elektrisches Signal Vorliegenden Daten in elektromagnetische Wellen umzuwandeln die sich dann durch die Umgebung ausbreiten können. Diese Signale können dann beim Empfänger wiederum in elektrische Signale umgewandelt und somit interpretiert werden. 

Für die drahtlose Telekommunikation werden überweigend elektromagnetische Wellen, insbesondere Funkwellen, eingesetzt. Drahtlose Kommunikationssysteme arbeiten in verschiedenen Frequenzbändern, die stark reguliert sind, um mögliche Interferenzen zu vermeiden. \autocite{GrundkenntnisseDrahtlosenKommunikation2023} 

Bekannte Modulationsverfahren lassen sich in Analoge und Digitale Verfahren aufteilen. Bei der analogen Modulation werden die Parameter des Trägersignals (Amplitude Modulation (AM) oder Frequenz Modulation (FM)) kontinuierlich entsprechend dem analogen Eingangssignal verändert. Bei der digitalen Modulation hingegen wird zwischen diskreten, fest definierten Zuständen umgeschaltet, um digitale Daten zu übertragen. \autocite[S. 112 ff. und S. 156 ff.]{ziemerPrinciplesCommunicationsSystems2015}

Ein einfaches Digitales Übertragungsprotokoll ist hier bei das Frequency Shift Keying (FSK), dabei werden digiale informationen durch die Variation der Frequenzen eines Trrägers kodiert. Im Wesentlichen wird die Trägerfrequenz periodisch zwischen mehreren Frequenzen verschoben, wobei jede Frequenz ein bestimmtes digitales Symbol darstellt.

Das einfachste FSK-Verfahren ist die binäre FSK (Binary FSK, BFSK oder 2-FSK), bei der zwei unterschiedliche Frequenzen verwendet werden, um die Binärziffern '0' und '1' zu repräsentieren. Wie in Beispielsweise Abbildung \ref{fig:frequency-shift-keying} kann eine höhere Frequenz eine binäre '1' darstellen, während eine niedrigere Frequenz eine binäre '0' repräsentiert. Wenn die zu übertragenden Daten eine '0' enthalten, wird die Trägerfrequenz $t_1$ verwendet um dieses Bit zu übertragen, wenn die Daten eine '1' sind, wird die Trägerfrequenz $t_2$ verwendet.\autocite{FrequencyShiftKeyingModulation2024}

\begin{figure}[H]
\centering
\includegraphics[scale=.5]{figures/asstes/frequency-shift-keying.png}
\caption{Frequency Shift Keying | Quelle: \autocite {FrequencyShiftKeyingModulation2024}}
\label{fig:frequency-shift-keying}
\end{figure}
\subsubsection*{Frequenz und Bandbreite}
Die Frequenz beschreibt, wie oft sich ein periodisches Signal pro Sekunde wiederholt. Sie wird in Hertz (Hz) angegeben und ist eng mit der Wellenlänge $\lambda$ verknüpft. Diese lässt sich aus der Ausbreitungsgeschwindigkeit $c$ des Signals und der Frequenz $f$ über $\lambda = c / f$ berechnen. Höhere Frequenzen entsprechen kürzeren Wellenlängen und erfordern in der Regel kleinere Antennenabmessungen.

Die Bandbreite eines Kommunikationskanals bezeichnet den Frequenzbereich, innerhalb dessen ein Signal mit akzeptabler Qualität übertragen werden kann. Sie wird ebenfalls in Hertz angegeben und ist sowohl durch physikalische Eigenschaften des Übertragungsmediums als auch durch elektronische Komponenten begrenzt. Die Bandbreite bestimmt zusammen mit der Signalqualität die maximal mögliche Datenrate. In vielen Kanälen beträgt sie nur einen Bruchteil der Mittenfrequenz, typischerweise zwischen 1 \% und 10 \% des Trägerfrequenzbereichs. \autocite[S. 4]{proakisDigitalCommunications2008}

\subsubsection*{Schlüsselmetriken der Verbindungsqualität: RSSI, SNR und PDR}
Zur Bewertung der Verbindungsqualität werden primär drei Schlüsselmetriken herangezogen: der Received Signal Strength Indicator (RSSI), das Signal-to-Noise Ratio (SNR) und die Packet Delivery Rate (PDR).

Der \textbf{RSSI} ist wie im Namen schon enthalten ein Indikator für das empfangene Signal. Er dient als relatives Maß und gibt die Stärke des Funksignals an. Er wird typischerweise in dBm ausgedrückt, wobei Werte näher an Null ein stärkeres empfangenes Signal bedeuten. Faktoren, die den RSSI hauptsächlich beeinflussen, sind der Pfadverlust oder der Antennengewinn. Zudem kommen auch noch weitere Verluste durch Kabel und Steckverbinder, für die Antenne, an Sender und Empfänger hinzu. Darüber hinaus kann der RSSI aufgrund von Signalkollisionen, externen Interferenzen und dem Nah-Fern-Effekt abnehmen. Als direktes Maß für die empfangene Leistung ist der RSSI eine entscheidende Eingangsgröße für viele Ausbreitungsmodelle. 

Oft als S/N bezeichnet, stellt das \textbf{SNR} das Verhältnis der empfangenen Signalleistung zum Umgebungsrauschpegel dar, üblicherweise in Dezibel (dB) ausgedrückt. Es ist eine kritische Metrik zur Bestimmung der Gesamtqualität des empfangenen Signals. \autocite{SignaltoNoiseRatioSNR2020} Das SNR wird mit der Formel \ref{eq:snr} berechnet.
\begin{equation}
\label{eq:snr}
SNR (dB) = P_{receivedSignal} (dBm) - P_{noise} (dBm)
\end{equation}


Auch als Packet Reception Rate (PRR) bezeichnet, quantifiziert die \textbf{PDR} den Prozentsatz der Datenpakete, die erfolgreich über eine Kommunikationsverbindung gesendet und empfangen werden. Sie dient als Maß für die Netzwerkkonsistenz und -leistung. Die PDR wird durch eine Kombination von Faktoren beeinflusst, darunter die Bandbreite, die Übertragungsdistanz und die zugrunde liegende Signalqualität (RSSI, SNR). Insbesondere niedrige RSSI-Werte, insbesondere unter einem bestimmten Schwellenwert (z. B. -120 dBm), können zu Paketkorruption oder vollständigem Verlust führen.

Während der RSSI eine Angabe der Rohsignalstärke liefert, bietet das SNR ein nuancierteres Maß für die Signalqualität relativ zum Rauschen. Letztendlich stellt die PDR die Ergebnisgröße dar, die die kombinierten Effekte von Signalstärke, Signalqualität und vorherrschenden Netzwerkbedingungen zusammenfasst. \autocite{RSSISNR}
\subsection{LoRa und LoraWAN}
Im Nachfolgenden soll die Funktechnologie auf der diese Arbeit aufbaut beschrieben werden. Long Range (LoRa) bildet dabei die Pysikalische schicht wärend das Long Range Wide Area Network (LoraWAN) die Netzwerkschicht übernimmt. Die genaue aufteilung kann der Abbildung \ref{fig:lora-lorawan-osi} entnommen werden.

\begin{figure}[H]
\centering
\includegraphics[scale=.4]{figures/diagrams/LoraWAN_OSI.png}
\caption{LoRa und LoRaWAN im OSI Schichtenmodell \ref{eckLoRaWANImDetail2019}}
\label{fig:lora-lorawan-osi}
\end{figure}

\subsubsection*{LoRa}

\paragraph*{Einordnung und Überblick}  
LoRa ist ein proprietäres und patentiertes drahtloses Übertragungsverfahren, das von der Semtech Corporation entwickelt wurde. 
Die Technologie arbeitet auf der physikalischen Schicht, auch Bitübertragungsschicht genannt, und basiert auf der Spread-Spectrum-Modulationstechnik \textit{Chirp Spread Spectrum} (CSS). 
Ziel von LoRa ist es, mit sehr geringem Energieverbrauch eine hohe Reichweite zu ermöglichen, auch unter schwierigen Ausbreitungsbedingungen. 

\paragraph*{Modulationstechnik}  
Chirp Spread Spectrum (CSS) ist ein Modulationsverfahren, bei dem die Frequenz des Signals während der Übertragungszeit kontinuierlich ansteigt oder abfällt.
Ein „Upchirp“ (Abbildung \ref{fig:lora-chirp} links) ist eine Erhöhung der Frequenz von niedrig nach hoch, während ein „Downchirp“ (Abbildung \ref{fig:lora-chirp} rechts) eine Absenkung der Frequenz von hoch nach niedrig darstellt. 

\begin{figure}[H]
\centering
\includegraphics[scale=.4]{figures/asstes/lora-chirps.png}
\caption{LoRa Chirps \cite{tulkaLoRaSpreadingFactor}}
\label{fig:lora-chirp}
\end{figure}

Durch diese Methode wird das Datensignal über ein breiteres Frequenzband verteilt, wodurch es robust gegenüber Rauschen und Interferenzen wird.  
Ein wesentlicher Vorteil von CSS gegenüber anderen Modulationstechniken, wie z.\,B. der in Abschnitt \ref{sec:drahtlosedatenübertragung} genannten FSK, ist die Fähigkeit, Signale selbst dann zu dekodieren, wenn sie unterhalb des Rauschpegels liegen.



\paragraph*{Symbolstruktur und Datenübertragung}  
LoRa überträgt Daten in Symbolen, die die kleinste Informationseinheit darstellen. Ein Symbol fasst eine bestimmte Anzahl an Bits zusammen, deren Anzahl durch den sogenannten Spreading Factor (SF) bestimmt wird. Je höher der SF, desto mehr Bits pro Symbol und desto größer die Anzahl an möglichen Symbolwerten. \\

Der Spreading Factor (SF) ist ein zentraler Parameter der LoRa-Modulation. Er beschreibt das Verhältnis zwischen Chiprate und Symbolrate und bestimmt damit direkt die Datenrate $R_b$ eines Systems.  

\begin{equation}
\label{eq:spredigfactorbandwith}
R_b = \frac{SF \cdot B}{2^{SF}} \cdot \frac{4}{4 + CR},
\end{equation}

Wobei $B$ die Bandbreite und $CR$ der Codierungsfaktor ist \cite[S.6]{rhode&schwarzCharacterizationLoRaDevices}.  

Mit steigendem SF verlängert sich die Dauer eines Symbols, da der zugrunde liegende Chirp über eine größere Zeitspanne ausgesendet wird. Dadurch wird die Übertragung robuster gegenüber Rauschen und Störungen und erlaubt größere Reichweiten. Gleichzeitig sinkt jedoch die Datenrate, da weniger Symbole pro Zeiteinheit übertragen werden können. In der Praxis bedeutet das: Ein niedriger SF (z.\,B. SF7) ermöglicht eine hohe Datenrate, eignet sich aber nur für kurze Distanzen mit gutem Empfang, während ein hoher SF (z.\,B. SF12) sehr große Reichweiten erlaubt, jedoch nur eine geringe Datenrate bereitstellt. Der Spreading Factor stellt somit einen zentralen Kompromiss zwischen Reichweite und Datenrate dar. \\

Die Übertragung erfolgt nicht durch feste Töne oder Frequenzen wie beispielsweise bei FSK, sondern über das CSS Verfahren. Bei LoRa steigt die Frequenz während der Symboldauer linear an (Upchirp). Der jeweilige Startpunkt des Chirps auf der Frequenzachse wird dabei zyklisch verschoben und codiert so den Symbolwert. Jedes Symbol entspricht also einem verschobenen Chirp (siehe Abbildung \ref{fig:lora-symbole}). 

\begin{figure}[H]
\centering
\includegraphics[scale=.4]{figures/asstes/n-LoRa-CSS-each-symbol-is-encoded-as-a-single-circularly-shifted-chirp-and-covers-the.png}
\caption{LoRa Symbole \cite{inproceedings}}
\label{fig:lora-symbole}
\end{figure} 

Beim Empfänger wird das Signal mit einem Referenz-Downchirp multipliziert. Dadurch wird aus dem verschobenen Chirp ein nahezu konstanter Ton, dessen Frequenz eindeutig dem gesendeten Symbolwert entspricht. Über eine diskrete Fourier-Transformation (DFT) lässt sich dieser Wert präzise bestimmen. So können auch Signale erkannt werden, die unterhalb des normalen Rauschpegels liegen. \autocite{tulkaLoRaSpreadingFactor, 8067462}

\paragraph*{Regulatorische Rahmenbedingungen (EU)}
In Europa wird das EU863–870-Band für LoRa verwendet mit \emph{Duty-Cycle}-Grenzen je Subband und Sendeleistungsgrenzen. Gerätekonfiguration und Kanalpläne folgen den \emph{LoRa Alliance Regional Parameters} (RP002), während die verbindlichen Funkparameter in \emph{ETSI EN~300~220} festgelegt sind. 
\autocite{RP002104, ETSIEN3002202025}

\paragraph*{Zusammenfassung}  
LoRa nutzt CSS als Modulationstechnik, wodurch eine robuste Datenübertragung selbst unter dem Rauschpegel möglich ist. 
Der Spreading Factor ist der zentrale Parameter, der Reichweite und Datenrate bestimmt. 
Die Datenübertragung erfolgt über symmetrische Chirp-Symbole, die durch Korrelation und Fourier-Analyse effizient erkannt werden können. 
Damit eignet sich LoRa besonders für Anwendungen mit geringer Datenrate, aber hoher Reichweitenanforderung. 


\subsubsection*{LoRaWAN} 
\label{sec:lorawan}

LPWANs (Low Power Wide Area Networks) ermöglichen energieeffiziente Kommunikation über große Distanzen und gelten daher als eine Schlüsseltechnologie für das Internet der Dinge (IoT). LoRaWAN zählt zu den vielversprechendsten LPWAN-Technologien. Es bietet geringe Leistungsaufnahme, niedrige Kosten und große Reichweite, geht jedoch mit einer geringen Datenrate einher. Während LoRa lediglich die physikalische Modulation beschreibt, definiert LoRaWAN (Long Range Wide Area Network) das darüberliegende Netzwerkprotokoll und die Architektur für die Kommunikation zwischen Endgeräten, Gateways und Netzwerkservern.

\paragraph*{LoRaWAN Netzwerkarchitektur}
Ein typisches LoRaWAN-Netzwerk besteht aus drei zentralen Komponenten. An erster Stelle stehen die Endgeräte, auch \textit{Nodes} genannt (in Abbildung \ref{fig:lorawan-architektur} unter \textit{End Nodes}). Dabei handelt es sich um Sensoren oder Aktoren, die Daten erfassen oder Befehle ausführen können. Diese Geräte besitzen keine direkte Internetanbindung, sondern lediglich einen LoRa-Chip sowie den LoRaWAN-Stack in ihrer Firmware. Alle Daten werden also über diese Schnittstelle empfangen oder gesendet.

Die zweite Komponente bilden die Gateways (in Abbildung \ref{fig:lorawan-architektur} unter \textit{Concentrator / Gateway}). Sie sind mit einem LoRa-Chip und einer aktiven Internetverbindung ausgestattet und fungieren als Schnittstelle zwischen der drahtlosen LoRa-Kommunikation und dem IP-Netzwerk. Sobald ein Gateway ein gültiges LoRaWAN-Paket empfängt, leitet es dieses an den Netzwerkserver weiter.

Der Netzwerkserver stellt die zentrale Instanz des gesamten LoRaWAN-Systems dar (in Abbildung \ref{fig:lorawan-architektur} unter \textit{Network Server}). Er filtert redundante Nachrichten, übernimmt die Authentifizierung und Sicherheit, steuert die Weiterleitung an die jeweiligen Anwendungsserver und verwaltet gleichzeitig die angebundenen Endgeräte \cite{LoRaWANArchitecture}.

\begin{figure}[H]
\centering
\includegraphics[scale=.135]{figures/asstes/lorawan-architecture.png}
\caption{LoRaWAN Architektur \cite{LoRaWANArchitecture}}
\label{fig:lorawan-architektur}
\end{figure} 

\paragraph*{LoRaWAN Geräteklassen}
LoRaWAN unterscheidet drei Gerätekategorien, die sich vor allem in ihren Energiesparmechanismen und Kommunikationsmöglichkeiten unterscheiden. Die energieeffizienteste Variante und zugleich die Mindestanforderung an ein LoRaWAN-Endgerät ist die Geräteklasse \textbf{A}. Sie eignet sich besonders für batteriebetriebene Systeme. Wie alle Klassen unterstützt auch Klasse A eine bidirektionale Kommunikation. Um den Energieverbrauch jedoch so gering wie möglich zu halten, können Downlink-Nachrichten vom Netzwerkserver nur kurz nach einem Uplink empfangen werden. Zu diesem Zweck öffnet das Endgerät nach jeder eigenen Übertragung zwei kurze Empfangsfenster (RX1 und RX2).

Die Geräteklasse \textbf{B} erweitert dieses Konzept um zusätzliche, periodische Empfangsfenster. Dadurch lassen sich Downlink-Nachrichten planbar zustellen. Jedes Gerät startet zunächst in Klasse A und kann vom Netzwerkserver in Klasse B versetzt werden, sofern es den entsprechenden Standard unterstützt.

Die dritte Kategorie ist die Geräteklasse \textbf{C}. Sie ist vor allem für netzbetriebene Geräte gedacht, da ihr Empfangsfenster (RX2) permanent geöffnet ist. Damit wird eine verzögerungsfreie Kommunikation zwischen Endgerät und Netzwerkserver möglich, was allerdings mit einem deutlich höheren Energieverbrauch verbunden ist \autocite{sornin2015lorawan}.

\paragraph*{LoRaWAN Join-Mechanismen}  
Damit ein Endgerät Nachrichten an ein LoRaWAN-Netzwerk senden kann, muss es sich zunächst anmelden. Hierfür existieren zwei grundlegende Verfahren: \textit{Over-The-Air Activation} (OTAA) und \textit{Activation by Personalization} (ABP).  

Bei OTAA, dem empfohlenen und sichersten Mechanismus, authentifiziert sich das Endgerät über einen Join-Vorgang beim Netzwerkserver. Das Gerät sendet dazu eine Join-Request-Nachricht, die unter anderem die eindeutige Gerätekennung \textit{DevEUI} sowie eine eindeutige Anwendungskennung enthält. In LoRaWAN v1.0.x wird diese als \textit{AppEUI} bezeichnet, während ab v1.1 der Begriff \textit{JoinEUI} verwendet wird. Zusätzlich wird ein \textit{DevNonce} übertragen, ein einmaliger Wert, der sicherstellt, dass Join-Nachrichten nicht wiederverwendet werden können, um Replay-Angriffe zu verhindern. Die Join-Nachricht ist zwar unverschlüsselt, jedoch mit einem MIC (Message Integrity Code) abgesichert.  

Nach erfolgreicher Prüfung sendet der Netzwerkserver eine Join-Accept-Nachricht zurück, die weitere Parameter wie \textit{NetID}, \textit{DevAddr} sowie Nonces für die Schlüsselerzeugung enthält. Anschließend werden die Sitzungsschlüssel abgeleitet. In LoRaWAN v1.0.x entstehen aus dem geheimen \textit{AppKey} zwei Session Keys, nämlich der \textit{NwkSKey} für die Netzwerkschicht und der \textit{AppSKey} für die Anwendungsschicht. Ab Version 1.1 unterscheidet LoRaWAN stärker zwischen Netzwerk- und Anwendungsebene. Hier wird der \textit{AppSKey} weiterhin aus dem \textit{AppKey} erzeugt, während drei verschiedene Netzwerkschlüssel (\textit{FNwkSIntKey}, \textit{SNwkSIntKey} und \textit{NwkSEncKey}) aus einem separaten \textit{NwkKey} abgeleitet werden. Durch diese Trennung wird die Sicherheit und Flexibilität bei der Schlüsselverwaltung erhöht.  

Im Gegensatz dazu verzichtet ABP vollständig auf einen Join-Prozess. Die für die Kommunikation erforderlichen Parameter, also die Geräteadresse \textit{DevAddr} sowie die Session Keys, werden hierbei direkt im Gerät vorkonfiguriert und auch im Netzwerkserver hinterlegt. In LoRaWAN v1.0.x handelt es sich dabei um den \textit{NwkSKey} und den \textit{AppSKey}, während ab Version 1.1 zusätzlich die drei Netzwerkschlüssel \textit{FNwkSIntKey}, \textit{SNwkSIntKey} und \textit{NwkSEncKey} berücksichtigt werden müssen. ABP ermöglicht somit eine sofortige Nutzung des Netzwerks, bietet jedoch geringere Sicherheit, da die Schlüssel statisch sind und nicht regelmäßig erneuert werden.  

Zusammenfassend bietet OTAA durch die dynamische Erzeugung von Sitzungsschlüsseln bei jedem Join deutlich mehr Sicherheit und Flexibilität, während ABP vor allem für Testaufbauten oder spezielle Szenarien mit festen Parametern genutzt wird.  

\paragraph*{Leistungsmerkmale und Herausforderungen}
LoRaWAN kann durch Parameteroptimierung an unterschiedliche Anwendungen angepasst werden. Wichtige Designaspekte sind Skalierbarkeit, Durchsatz, Abdeckung, Energieeffizienz und geringe Kosten \cite{bor2017lora}. Herausforderungen bestehen insbesondere in:

\begin{itemize}
    \item \textbf{Skalierbarkeit:} Beeinflusst durch Faktoren wie verfügbare Kanäle, Spreizfaktor, Bandbreite und regulatorische Einschränkungen. In Europa steht für LoRa das Band von 863 MHz bis 870 MHz zur Verfügung, in Amerika von 902 MHz bis 928 MHz \autocite{FrequencyPlans}.
    
    \item \textbf{Energieeinsparung:} Durch die Adaptive Data Rate (ADR) oder eine optimierte Parameterwahl kann der Energieverbrauch zusätzlich reduziert werden \autocite{kufakunesuSurveyAdaptiveData2020}.
    
    \item \textbf{Sicherheit:} LoRaWAN setzt auf Ende-zu-Ende-Verschlüsselung mit einer klaren Aufgabentrennung zwischen Join Server, Network Server und Application Server. Damit werden Schlüssel sicher verwaltet, Betreiber erhalten nur die für sie notwendigen Informationen, und Multi-Tenant-Szenarien werden unterstützt. Neuere Spezifikationen (ab v1.1) verbessern zudem die Schlüsselverwaltung durch die Trennung von Anwendungs- und Netzwerkschlüsseln \autocite{butun2019security,LoRaWANBackendInterfaces11}.
\end{itemize}

\paragraph*{Anwendungsgebiete}
LoRaWAN eignet sich für zahlreiche IoT-Szenarien, darunter smarte digitale Städte, smarte Messungen (z.\,B. für Pflanzen), intelligentes Parkraummanagement oder adaptive Straßenbeleuchtung \autocite{BadenWuerttembergFoerdertLong2024}. Aufgrund der niedrigen Bitrate ist der Einsatz jedoch auf Anwendungen mit geringer Datenübertragungsrate beschränkt.

\paragraph*{Roaming, Peering und Interoperabilität}
LoRaWAN-Roaming erlaubt es, dass Endgeräte auch außerhalb des eigenen Netzwerks funktionieren können. Es gibt zwei Arten von Roaming:

\begin{itemize}
  \item \textbf{Passives Roaming:} Das Endgerät bleibt unter der Kontrolle seines Heimat-Netzservers (Serving NS), die Funkverbindung läuft aber über fremde Gateways und Netzserver (Forwarding NS). Diese leiten die Daten nur weiter, ohne die eigentliche Gerätesteuerung zu übernehmen.
  \item \textbf{Handover Roaming:} Hier übergibt der Heimat-Netzserver die Kontrolle an einen besuchten Netzserver (Serving NS). Dieser steuert dann das Endgerät aktiv, während der Heimat-NS im Hintergrund für die Schlüsselaushandlung und Verwaltung eingebunden bleibt.
\end{itemize}

Damit Roaming überhaupt funktioniert, müssen Netzbetreiber Peering- oder Roamingabkommen schließen. 
\emph{Packet Broker} bietet dazu ein globales Vermittlungsnetz, das den Austausch von Paketen zwischen verschiedenen Betreibern erleichtert. Ein Beispiel ist das offene Netz \texttt{The Things Network (TTN)} mit der \texttt{NetID~000013}. 

Kommerzielle Anbieter wie Senet setzen zusätzlich auf bilaterale Roaming-Vereinbarungen mit Partnernetzen, um ihre Abdeckung zu erweitern. \cite{LoRaWANBackendInterfaces11,PacketBroker,TTNNetID,SenetExt}
\subsection{Öffentliche LoraWAN Netzwerke}
\label{sec:ttn}

Öffentliche LoRaWAN-Netze stellen Konnektivität als gemeinschaftliche oder kommerzielle Infrastruktur bereit. Im Gegensatz zu privaten Netzen (unter eigener Kontrolle) unterscheiden sie sich durch \emph{Zugang} (offen vs. vertraglich), \emph{Kostenmodell} (kostenfrei/Fair-Use vs. verbrauchs- oder Service-Level-Agreement (SLA)-basiert \footnote{Ein SLA ist eine vertraglich festgelegte Leistungszusage zwischen Anbieter und Nutzer eines Dienstes (z.\,B. garantierte Verfügbarkeit, Reaktionszeit oder Durchsatz). \emph{SLA-basiert} bedeutet, dass das Netzwerk diese zugesicherten Leistungsmerkmale vertraglich garantiert.}), \emph{Peering/Roaming} sowie \emph{Betriebs- und Sicherheitsprozesse}. 
\autocite{LoRaWANBackendInterfaces11, RP002104, ETSIEN3002202025}

\subsubsection*{Abgrenzung und Kategorien}
In der Praxis zeigen sich drei Netzwerktypen:

\begin{enumerate}
  \item \textbf{Community-Netze} (z.\,B. The Things Network, TTN): offen, best-effort, Fair-Use, Enterprise-Plan.
  \item \textbf{Dezentrale Crowd-Netze} (z.\,B. Helium~IoT): von vielen Hotspot-Betreibern getragen, nutzungsbasierte Abrechnung.
  \item \textbf{Carrier/Neutral-Host} (z.\,B. Senet, Everynet): vertraglich, SLA/Peering, nur nationale Abdeckung.
\end{enumerate}

\subsubsection*{The Things Network (TTN)}
TTN ist ein globales Community-Netz auf Basis von \emph{The Things Stack}. Es bietet einen kostenfreien Zugang unter einer Fair-Use-Policy. Die Fair-Use-Policy limitiert in der kostenfreien Version u.\,a. die Uplink-Airtime pro Gerät und Tag. Genauer sind es „30 Sekunden Uplink-Airtime pro Tag (24 Stunden) und Gerät sowie maximal 10 Downlink-Nachrichten pro Tag (24 Stunden) und Gerät“ \autocite{DutyCycleTTN}. SLAs bestehen im Community-Netz nicht, jedoch bietet die Enterprise-Variante kommerzielle Zusagen. Das Netzwerk nutzt \emph{Packet Broker} als globales Rückrad zur sicheren Weiterleitung von Verkehr zwischen Netzen. Das TTN ist über die Community-Projekte TTN Mapper \& Packet-Broker-Mapper kartiert \autocite{TTNFairUse, PacketBroker, TTNMapperDoc}.  

Die Motivation zur Aufstellung eigener Gateways liegt im gemeinschaftlichen Charakter des Netzes: Durch zusätzliche Gateways wird die Netzabdeckung insgesamt verbessert, wovon wiederum alle Teilnehmer profitieren. Auch der Betreiber selbst erhält dadurch typischerweise eine stabilere Verbindung und bessere Abdeckung für seine eigenen Geräte. Jedoch ohne dass er dadurch mehr Nutzungsrechte erhält. TTN versteht sich damit als offenes, gemeinschaftsgetriebenes Infrastrukturprojekt, das vor allem durch den Nutzen der gemeinsamen Netzressourcen getragen wird.

\subsubsection*{Helium IoT}
Helium IoT ist ein globales LoRaWAN-Netzwerk, das von vielen unabhängigen Teilnehmern betrieben wird. Im Unterschied zu TTN ist die Nutzung nicht kostenlos, sondern erfolgt über sogenannte \emph{Data Credits (DC)}. Diese sind an den US-Dollar gebunden (fester Wert: 1\,DC = 0{,}00001\,USD) und können nur durch das \emph{Burning} von HNT-Token erzeugt werden. Dadurch wird ein direkter ökonomischer Anreiz geschaffen, der die Nachfrage nach Netzwerkressourcen mit dem Helium-Krypto-Token verbindet \autocite{HeliumDC}. 

Ein Uplink von 24~Byte entspricht einem Abrechnungsblock. Downlinks sind in der Regel kostenfrei, können jedoch durch zusätzliche Konfiguration zuverlässiger gestaltet werden. Dafür können Betreiber von LoRaWAN-Network-Servern (LNS) mehrere Gateway-Duplikate pro Uplink \enquote{einkaufen} (\emph{multibuy}), um die Wahrscheinlichkeit einer erfolgreichen Zustellung zu erhöhen \autocite{HeliumLNSAdv}.  

Neben der Nutzung führt das Modell von Helium auch zu einem finanziellen Anreiz für die Aufstellung von Gateways: Ähnlich wie bei anderen Kryptowährungen wird einem Betreiber eines Miners ein wirtschaftlicher Nutzen in Form von Token-Anteilen zugesichert. Betreiber erhalten durch die Bereitstellung eines Gateways und der damit verbundenen Netzwerkabdeckung Vergütungen in Form von HNT-Tokens. Die Höhe dieser Vergütung hängt dabei maßgeblich von der Erreichbarkeit benachbarter Gateways sowie von der Gateway-Dichte in der jeweiligen Region ab. Dieses Belohnungsmodell führte in der Vergangenheit zu einem schnellen und globalen Ausbau der Netzabdeckung, jedoch mit einer stark variierenden Qualität von Hardware und Installationsbedingungen.  

Während das Helium-Netz dadurch eine breite Abdeckung erreichen konnte, ist die resultierende Infrastruktur weniger einheitlich als bei klassischen Community-Netzen. Professionelle Gateways mit optimaler Standortwahl stehen oftmals neben kostengünstigen Geräten, die unter suboptimalen Bedingungen betrieben werden. Dies wirkt sich unmittelbar auf die Netzqualität und Zuverlässigkeit aus, verdeutlicht jedoch zugleich die Stärke des ökonomischen Anreizes als Treiber für den Netzausbau.

Die Netzabdeckung wird u.\,a. durch \emph{CoverageMap.net} vermessen, ein Service, der ursprünglich aus dem TTN Mapper-Projekt hervorgegangen ist \autocite{TTNMapperDoc}.


\subsubsection*{Carrier-/Neutral-Host-Netze (Beispiele: Senet, Everynet, Actility/ThingPark)}
Carrier- bzw. Neutral-Host-Netze bieten Konnektivität auf Basis von vertraglich zugesicherten Service-Level-Agreements (SLAs) an und greifen dabei teilweise auf überregionale Roaming-Abkommen zurück. 
Ein Beispiel ist \emph{Senet} in den USA, das durch Roaming-Abkommen eine erweiterte Netzabdeckung („Extended Coverage“) bereitstellt, unter anderem auch durch eine Integration mit Helium \autocite{SenetExt}. 
Das Unternehmen \emph{Everynet} betreibt Neutral-Host-Netze in mehreren Ländern und bietet zusätzlich Integrationen mit \emph{AWS IoT Core for LoRaWAN} an \autocite{AWSPublic}. 
\emph{Actility} mit seiner Plattform \emph{ThingPark} gilt als einer der verbreitetsten Anbieter in diesem Bereich und stellt sowohl Roaming-Funktionalitäten als auch hybride Szenarien für verschiedene Betreiber bereit \autocite{ETSIEN3002202025}.


\subsubsection*{Praktische Einordnung für diese Arbeit}
Für prototypische \emph{Tracking}-Workloads sind TTN (kostenfrei, Fair-Use) und Helium (breite Crowd-Abdeckung, DC-basiert) aufgrund ihrer großen Abdeckung und einfachen Anwendung naheliegend. 
Während TTN durch seine Community-getragene Struktur häufig qualitativ hochwertige Gateways bietet, die von technisch versierten Betreibern (z.\,B. Universitäten oder Kommunen) installiert werden, zeichnet sich Helium durch eine sehr große Anzahl an privat betriebenen Hotspots aus. 
Dies führt zu einer breiten globalen Abdeckung, jedoch erreichen die im Helium-Netz eingesetzten Gateways häufig nicht die Qualitätsstandards, wie sie im TTN-Umfeld üblich sind. Gründe hierfür sind unter anderem weniger optimale Installationsorte sowie der Einsatz kostengünstigerer Hardware, beispielsweise bei Antennen und Kabeln.
TTN eignet sich somit besonders für stabile Prototyping-Szenarien mit gut dokumentierten Tools und Community-Support, während Helium Vorteile bei einer schnellen globalen Skalierbarkeit bietet. 

\subsection{LoRaWAN Industrielle Anwendungsbereiche}
LoRaWAN (Long Range Wide Area Network) bietet durch seine Charakteristiken die in Abschnitt \ref{sec:lorawan} beschrieben viele Vorteile für industrielle Anwendungen. 

\paragraph*{Vorausschauende Instandhaltung} 
LoRaWAN ermöglicht eine zuverlässige und energiearme Übertragung von Zustandsdaten (z. B. Vibration, Temperatur) die für frühzeitige Wartungserkennung genutzt werden können. Besonders im Industrie-4.0-Umfeld lassen sich Sensoren schnell installieren, mit Legacy-Geräten koppeln und in Cloud-basierte Analyseplattformen integrieren, um Effizienz, Sicherheit und Zuverlässigkeit zu steigern \autocite{lorawansmartindustry}.

\paragraph*{Telemetrie und Alarmüberwachung}
LoRaWAN eignet sich für die gleichzeitige Übertragung von periodischen Telemetriedaten und seltenen Alarmmeldungen. Durch unterschiedliche Spreading-Factor-Zuweisung und gezielte Retransmission kann hohe Zuverlässigkeit für Alarme bei minimaler Verzögerung erreicht werden—ohne die reguläre Kommunikation zu beeinträchtigen \autocite{santos2020}.

\paragraph*{Umwelt- und Gefahrenüberwachung}
In industriellen Umgebungen mit Unfall- oder Störfallrisiko (z. B. toxische Gase) erlaubt ein LoRaWAN-System mit optimiertem Downlink-Control-Packet-Mechanismus zuverlässige, latenzarme Umweltdatenübermittlung zur schnellen Gefahrenabwehr \autocite{tamang2022}.

\subsection{Verwandte Arbeiten}
In diesem Abschnitt wird der aktuelle Forschungs- und Entwicklungsstand zu LoRaWAN im Kontext von Tracking-Systemen dargestellt. Ziel ist es, existierende Studien und Projekte zu analysieren, die sich mit der Leistungsfähigkeit, Energieeffizienz und Praxistauglichkeit von LoRaWAN im Bereich der globalen Güterüberwachung beschäftigen. 

\paragraph*{LoRaWAN im Vergleich zu anderen LPWAN-Technologien}
Mehrere Arbeiten vergleichen LoRaWAN mit alternativen LPWAN-Technologien wie Sigfox, NB-IoT und LTE-M. Studien zeigen, dass LoRaWAN insbesondere durch seine Flexibilität und die Verfügbarkeit von Community-Netzen (z.\,B. The Things Network genauer in Abschnitt \ref{sec:ttn} beschrieben) oft einfacher zu verwenden ist, während NB-IoT Vorteile in Netzstabilität und QoS bietet dafür aber auf Lizensierten Frequenzbändern arbeiten \cite{mekki2019overview, centenaro2016long,adelantado2017understanding}.  

\paragraph*{Leistungsfähigkeit und Reichweite}
Analysen der Reichweite von LoRaWAN zeigen, dass in urbanen Umgebungen typische Reichweiten von 2–5 km und in ländlichen Gebieten bis zu 15 km erreicht werden können \cite{augustin2016study, petajajarvi2017coverage}. Faktoren wie Bebauung, Antennenhöhe und Spreading Factor haben signifikanten Einfluss auf die Verbindungsqualität. 

\paragraph*{Energieeffizienz}
Für batteriebetriebene Endgeräte ist die Energiebilanz entscheidend. Arbeiten wie \cite{fialhoBatteryLifetimeEstimation2020, raza2017low,qiu2018survey} zeigen, dass LoRaWAN durch die Class-A-Architektur eine hohe Energieeffizienz erreicht, die Laufzeiten von mehreren Jahren bei typischen Sensordatenraten ermöglicht. Allerdings erfordert die Konfiguration des Duty-Cycles und der Sendeintervallen eine Balance zwischen Datenaktualität und Batterielaufzeit.

\paragraph*{Anwendungsgebiete in der Logistik}
LoRaWAN wird in zahlreichen Szenarien des IOTs eingesetzt, insbesondere in der Logistik für Güterverfolgung und Anwendungen in intelligenten Städten. Charakteristisch für die Anwendungsszenarien sind niedrige Datenraten, extrem geringer Energieverbrauch und lange Batterielaufzeiten, was für großflächige, wartungsarme Sensornetze wichtig ist. Im Bereich der Logistik werden vor allem Anwendungen wie die Überwachung von Transportgütern, Temperatur- oder Positionsdaten und allgemeine Güterverfolgung adressiert \cite{qiu2018survey}. Darüber hinaus betonen beide Studien die Bedeutung von Erweiterbarkeit und Zusammenarbeit zwischen verschiedenen Netzwerken, was durch Mechanismen wie Netzwechsel unterstützt werden kann \cite{raza2017low}.
