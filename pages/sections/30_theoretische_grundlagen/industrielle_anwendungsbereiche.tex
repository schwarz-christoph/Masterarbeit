LoRaWAN (Long Range Wide Area Network) bietet durch seine Charakteristiken die in Abschnitt \ref{sec:lorawan} beschrieben viele Vorteile für industrielle Anwendungen. 

\paragraph*{Vorausschauende Instandhaltung} 
LoRaWAN ermöglicht eine zuverlässige und energiearme Übertragung von Zustandsdaten (z. B. Vibration, Temperatur) die für frühzeitige Wartungserkennung genutzt werden können. Besonders im Industrie-4.0-Umfeld lassen sich Sensoren schnell installieren, mit Legacy-Geräten koppeln und in Cloud-basierte Analyseplattformen integrieren, um Effizienz, Sicherheit und Zuverlässigkeit zu steigern \autocite{lorawansmartindustry}.

\paragraph*{Telemetrie und Alarmüberwachung}
LoRaWAN eignet sich für die gleichzeitige Übertragung von periodischen Telemetriedaten und seltenen Alarmmeldungen. Durch unterschiedliche Spreading-Factor-Zuweisung und gezielte Retransmission kann hohe Zuverlässigkeit für Alarme bei minimaler Verzögerung erreicht werden—ohne die reguläre Kommunikation zu beeinträchtigen \autocite{santos2020}.

\paragraph*{Umwelt- und Gefahrenüberwachung}
In industriellen Umgebungen mit Unfall- oder Störfallrisiko (z. B. toxische Gase) erlaubt ein LoRaWAN-System mit optimiertem Downlink-Control-Packet-Mechanismus zuverlässige, latenzarme Umweltdatenübermittlung zur schnellen Gefahrenabwehr \autocite{tamang2022}.
