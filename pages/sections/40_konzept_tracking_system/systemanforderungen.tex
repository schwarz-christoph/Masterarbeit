Die Entwicklung des Tracking-Systems erfolgte unter der Prämisse, zunächst einen funktionsfähigen Prototypen als Proof-of-Concept zu realisieren. Dabei standen Aspekte wie Kostenoptimierung oder Robustheit im industriellen Einsatz nicht im Vordergrund. Stattdessen lag der Schwerpunkt auf Modularität und Erweiterbarkeit, um eine Plattform zu schaffen, die flexibel an unterschiedliche Szenarien angepasst werden kann. 

Als Basiskomponenten wurden eine GPS-Einheit zur Positionsbestimmung sowie eine SD-Karte zur lokalen Zwischenspeicherung vorgesehen. Externe Sensoren, beispielsweise zur Temperaturerfassung oder zur Anbindung von Sicherheitssiegeln (elektronische Plomben), sollten über modulare Schnittstellen integriert werden können. Dadurch ergibt sich eine universelle Plattform, die sich für unterschiedliche Anforderungen in der Transportlogistik anpassen lässt. 

Für die Kommunikation wurde der Einsatz von LoRaWAN spezifiziert. Aufgrund der Fokussierung auf energieeffiziente Uplink-Übertragungen ohne Bedarf an kontinuierlichen Downlinks wurde die Geräteklasse~A gewählt. Diese stellt die grundlegende und am weitesten verbreitete Gerätekategorie dar, die insbesondere für batteriebetriebene Endgeräte geeignet ist. Eine Bindung an ein spezifisches öffentliches oder privates Netzwerk wurde im Rahmen der Konzeption nicht vorgenommen, um die Flexibilität in der prototypischen Umsetzung zu wahren. 

Neben der Modularität wurde auch der Energieverbrauch als zentrale Anforderung berücksichtigt. Zwar war die Laufzeitoptimierung nicht vorrangiges Ziel der ersten Implementierung, dennoch sollte die Architektur grundsätzlich energieeffiziente Betriebsmodi unterstützen. Ergänzend dazu bietet LoRaWAN inhärente Sicherheitsmechanismen wie Ende-zu-Ende-Verschlüsselung und Schlüsselverwaltung, die für die angestrebten Einsatzszenarien als ausreichend betrachtet wurden. 
