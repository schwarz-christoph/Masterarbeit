Das Tracking Gerät hat zwei grundlegende Betriebsmodi die im Nachfolgenden kurz erläuchtert werden sollen:

\begin{itemize}
    \item \textbf{Periodisches Senden:} 
    Daten werden in festen Zeit Intervallen übermittelt (z.\,B. alle 5, 15 oder 60 Minuten). 
    Dieser Ansatz ist einfach und sorgt für eine kontinuierliche Nachverfolgbarkeit über die Transportstrecke. 
    Allerdings steigt der Energieverbrauch deutlich mit kürzer werdenden Intervallen \cite{lansitec2024lorawan,cloudstudio2023lorawan}.
    
    \item \textbf{Ereignisgesteuertes Senden:} 
    Eine Übertragung erfolgt nur bei bestimmten Ereignissen, etwa beim Öffnen einer Plombe, Überschreiten eines Temperaturgrenzwertes 
    oder bei Erschütterung des Transportguts. 
    Dadurch lässt sich die Batterielaufzeit zwar signifikant verlängern, jedoch sind die Positionsdaten zwischen den Ereignissen lückenhaft 
    \cite{tektelic2023assettracking,zhang2022lorawanmac}.
\end{itemize}

Am besten geeignet ist desswegen ein Hybrieder Ansatzt. Dabei sollen regelmäßige Intervall-Updates mit großen Intervallen mit zusätzlichen Ereignisübertragungen kombiniert werden. 
Dadurch kann eine Balance zwischen Energieeffizienz und Transparenz der Transportüberwachung erreicht werden.