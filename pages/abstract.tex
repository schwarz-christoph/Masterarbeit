\begin{abstract}

\section*{Zusammenfassung}
In dieser Bachelorarbeit wird die Implementierung und Integration von Continuos Deployment (CD) in die Software-Toolchain der MAN Truck \& Bus SE erörtert. Dabei liegt der Fokus auf Linux und cloudbasierten Lösungen. Hierbei sollen in der späteren Lösung Debian-Pakete deployt werden. Es werden gängige Deploymentmethoden miteinander verglichen und für den konkreten Anwendungsfall bewertet. Durch den Einsatz moderner Technologien wie Kubernetes, FluxCD und GitLab Runner wird aufgezeigt, wie der Deployment-Prozess automatisiert werden kann, um dadurch Fehler und Inkonsistenzen effektiv eliminieren zu können. Ebenso kann mithilfe eines autonomen Deploymentprozesses Zeit gespart werden, da keine manuellen Eingriffe mehr nötig sind. Besonders die Einführung von Versionierung der einzelnen Teile spielt eine entscheidende Rolle, in dem somit sichergestellt wird, dass Änderungen nachvollziehbar und kontrolliert erfolgen können. Ebenso wird die nötige Hardware für die Umsetzung diskutiert. Das entwickelte System bietet ein flexibles und erweiterbares Grundgerüst, das nicht nur für aktuelle, sondern auch für zukünftige Projekte der MAN genutzt werden kann.

\section*{Abstract}
This bachelor's thesis discusses the implementation and integration of Continuous Deployment (CD) into the software toolchain of MAN Truck \& Bus SE, focusing on Linux and cloud-based solutions. The future solution aims to deploy Debian packages. It compares common deployment methods and evaluates them for the specific application case. By leveraging modern technologies like Kubernetes, FluxCD, and GitLab Runner, it demonstrates how to automate the deployment process to effectively eliminate errors and inconsistencies. An autonomous deployment process saves time as manual interventions are no longer necessary. The introduction of versioning plays a crucial role by ensuring changes are traceable and controlled. The necessary hardware for implementation is also discussed. The developed system offers a flexible and expandable framework that can be used not only for current but also for future projects at MAN.


\end{abstract}